\documentclass{beamer}
\usetheme{Madrid}
\usefonttheme{professionalfonts}
\usecolortheme{rose}
\usepackage{fontspec}
\setmainfont{Times New Roman}
\usepackage{bm}
\usepackage{tikz}
\usepackage{svg}
\usepackage{graphics}
\usepackage{appendixnumberbeamer}
\usepackage{mathtools}
\usepackage{enumitem}
\setlist[itemize]{label=\textbullet}
\usetikzlibrary{decorations.pathreplacing}
\usetikzlibrary{arrows.meta}
\setlength{\parskip}{0.3em}
\AtBeginSection[]{
\begin{frame} 
    \frametitle{Contents}
    \tableofcontents[currentsection]
\end{frame}
}
\title[Amplication of BCFW]{A complete solution for scattering in a kind of quiver gauge theory}
\author{Su Yingze}
\institute{Nagoya University}
\date[4 21st 2025]{April 21st 2025}

\begin{document}
\frame{\titlepage}
\section{Preliminary}
\begin{frame}
    \frametitle{A brief introduction to BCFW}
    BCFW recursion relation is a method to compute scattering amplitude, especially in Yang-Mills theory and gravity.
    \par
 \begin{itemize}[label=\textbullet]
    \item Ruth Britto
    \item Freddy Cachazo
    \item Bo Feng
    \item Edward Witten
 \end{itemize}
\end{frame}
\begin{frame}
    \frametitle{From real to complex -- Analytic Continuation}
    \textbf{Why is analytic continuation valid?}
\begin{itemize}
  \item Tree level scattering amplitudes are rational functions of Lorentz invariants, such as $\bm{p_{i\mu}p_j^\mu}$, $\bm{p_{i\mu}\epsilon_j^\mu}$.
  \item \textbf{Locality} tells us that any pole of a tree-level amplitude must correspond to a on-shell propagating particle. 
  \item There's only single pole, no branch cuts (logs, square roots, etc) at tree level.
\end{itemize}
    \begin{center}
    \tikz{
      \draw[double equal sign distance, -Implies, line width=1pt] (0,0) -- (0,-1.2);
    }
    \end{center}
\vspace{0.5em}
\centering
\textcolor{red}{Ampltudes can be shifted to complex plane}
\end{frame}

\begin{frame}
    \frametitle{Momentum Shift in BCFW}
    \textbf{What did BCFW do to make the shift?}


    Here we consider the case in which all particles are massless, $p_i^2 = 0$ for all $i = 1, 2, \dotsc, n$. Then introduce $n$ complex-valued vectors $r_i^\mu$.
    \begin{enumerate}[label=(\roman*)]
        \item $\sum_{i=1}^n r_i^\mu=0$,
        \item $r_i\cdot r_j = 0$ for all $i,j=1, 2, \ldots, n.$ In particular $r_i^2=0$,
        \item $p_i \cdot r_i =0$ for each i (no sum). 
    \end{enumerate}

These vectors $r_i$ are used to define n shifted momenta
\begin{equation*}
    \hat{p}_i^\mu \equiv p_i^\mu + zr_i^\mu \qquad \text{with} z \in \mathcal{C}
\end{equation*}

\end{frame}
\begin{frame}
Note that,
    \begin{enumerate}[label=(\Alph*)]
    \item By property (i), momentum conservation holds for the shifted momenta: $\sum_{i=1}^{n} \hat{p}_i^\mu =0$,
    \item By (ii) and (iii), we have $\hat{p}_i^2=0$, so each shifted momentum is on-shell,
    \item For a non-trival subset of generic momenta $\{p_i\}_{i\in I}$, define $P_I^\mu=\sum_{i\in I}p_i^\mu$.
    \end{enumerate}
Then, $\hat{P}_I^2$ is \textcolor{red}{linear} in z:
\begin{equation*}
    \hat{P}_I^2=\left(\sum_{i\in I} \hat{p}_i \right) ^2 = P_I^2 +2zP_I\cdot R_I \quad \text{with} \quad R_I=\sum_{i\in I} r_i ,
\end{equation*}
because the $z^2$ term vanishes by property (ii). We can write 
\begin{equation*}
    \hat{P}_I^2 = -\frac{P_I^2}{z_I}(z-z_I) \quad \text{with} \quad z_I=-\frac{P_I^2}{2P_I\cdot R_I}
\end{equation*}
\end{frame}
\begin{frame}
    \frametitle{Fantasitic result from Cauchy Theorem}
    As a result of (A) and (B) (momentum conservation and on-shell), we can consider amplitude $A_n$ in terms of shifted momentum $\hat{p}_i^\mu$ instead of
    original real momentum. 
    \begin{equation*}
        A_n \longrightarrow \hat{A}_n(z)
    \end{equation*}
    and we have known the possible positions of single poles, $z_I$, different propagators give 
    us different single poles in the z-plane. 
    \par
    \textcolor{red}{$\bigstar$ The most important point is}
    \begin{equation*}
        \boxed{\color{red}A_n=\hat{A}_n(0)}
    \end{equation*}
\end{frame}




\end{document}