\documentclass{beamer}
\usetheme{metropolis}
% 防止目录变灰或消失
\setbeamercolor{section in toc}{fg=black,bg=white}
\setbeamercolor{section in toc shaded}{fg=black,bg=white}
\setbeamercolor{subsection in toc shaded}{fg=black,bg=white}
\setbeamercolor{alerted text}{fg=black}

% 禁用 shaded 效果
\setbeamertemplate{section in toc shaded}[default]
\setbeamertemplate{subsection in toc shaded}[default]

% ========= 字体 & 数学 =========
\usepackage{newtxtext,newtxmath}

% ========= 图像 & 物理包 =========
\usepackage{graphicx}
\usepackage{amsmath, amssymb}
\usepackage{tikz-feynman}
\usepackage{physics}


% ========= 修复 Metropolis 的变灰问题 =========


% ========= 标题信息 =========
\title{On-Shell Methods for Tree-Level Amplitudes\\ in (De)Constructed Gauge Theory}
\subtitle{Application of BCFW Recursion Relation}

\author[Su Yingze]{
  Su Yingze\\
  Supervisor: Prof. Tanabashi Masaharu
}

\institute{Theoretical Particle Physics Laboratory, Nagoya University}
\date{\today}

% ========= 标题页右下角校徽 =========
\titlegraphic{
  \vspace*{4.5cm}
  \hfill
  \includegraphics[width=2.5cm]{Nagoya.png}
  \vspace*{-2cm}
}

% ========= 正文开始 =========
\begin{document}

% ===== 封面 =====
\maketitle

% ===== 总目录页 =====
\begin{frame}{Outline}
  \setbeamertemplate{section in toc}[sections numbered]
  \tableofcontents[currentsection]
\end{frame}

% ===== 每章开始时显示当前目录页(可选)=====


% ===== 1. Motivation =====
\section{Motivation}
\begin{frame}{Why We Study Scattering Amplitudes?}
  \begin{enumerate}
    \item \textbf{Bridge between theory and experiment}
    \begin{itemize}
      \item Core prediction targets for high-energy experiments such as the LHC
      \item Any new theory (SUSY, GUTs, extra dimensions) must predict observable cross sections
    \end{itemize}
    \pause
    \item \textbf{Reveal deep structures of quantum field theory}
    \begin{itemize}
      \item Amplitudes exhibit hidden symmetries (e.g., dual conformal, Yangian) not visible in the Lagrangian
      \item These symmetries suggest deeper theoretical frameworks, such as integrability or AdS/CFT correspondence
    \end{itemize}
  \end{enumerate}
\end{frame}
\begin{frame}
    \frametitle{Struggles of Traditional Approaches}
  
\end{frame}

% ===== 2. Model =====
\section{Model}
\begin{frame}{(De)constructed Gauge Theory}
  \begin{itemize}
    \item Based on discretizing extra dimensions
    \item Multiple gauge groups + link scalar fields
    \item Focus: a general $n$-site structure and 2-site limit
  \end{itemize}
\end{frame}

% ===== 3. Methods =====
\section{Methods} 
\begin{frame}{Modern Amplitude Techniques}
  \begin{itemize}
    \item BCFW recursion: complex deformation of momenta
    \item Spinor-helicity formalism: simplifies massless amplitudes
    \item CSW expansion: handles NMHV amplitudes
  \end{itemize}
  \pause
  \[
  A_n = \sum_{\text{partitions}} \frac{A_L A_R}{P^2}
  \]
\end{frame}

% ===== 4. Results =====
\section{Results}
\begin{frame}{Key Results}
  \begin{itemize}
    \item Computed tree-level amplitudes for various helicity configurations
    \item Verified KK-mode cancellation patterns
    \item Demonstrated efficiency over traditional Feynman diagrams
  \end{itemize}
\end{frame}

% ===== 5. Outlook =====
\section{Outlook}
\begin{frame}{Future Directions}
  \begin{itemize}
    \item Extension to $n$-site models and loop-level amplitudes
    \item Application to black hole scattering (on-shell gravity)
    \item Exploration of geometric frameworks (amplituhedron, CHY)
  \end{itemize}
\end{frame}

% ===== 致谢 =====
\begin{frame}{Acknowledgement}
  \centering
  Thanks to my advisor, lab colleagues, and audience!
\end{frame}

\end{document}
