\documentclass[a4paper,12pt]{article}
\usepackage[margin=1in]{geometry}
\title{}
\author{}
\begin{document}
    \begin{itemize}
        \item %\cite{Arkani-Hamed:2019ymq}
        N.~Arkani-Hamed, Y.~t.~Huang and D.~O'Connell,
        ``Kerr black holes as elementary particles,''
        JHEP \textbf{01}, 046 (2020)
        doi:10.1007/JHEP01(2020)046
        [arXiv:1906.10100 [hep-th]].
        224 citations counted in INSPIRE as of 13 Sep 2024
    \end{itemize}
\section*{Hard to be catagorized}
\begin{itemize}
    \item %\cite{Aoude:2020onz}
    R.~Aoude, K.~Haddad and A.~Helset,
    ``On-shell heavy particle effective theories,''
    JHEP \textbf{05}, 051 (2020)
    doi:10.1007/JHEP05(2020)051
    [arXiv:2001.09164 [hep-th]].
    163 citations counted in INSPIRE as of 16 Sep 2024\\
    The author introduce on-shell variables for HPETs with the aim to present such a formalism will allow the extension of
    HBET to higher spins and to facilitate its application to higher loop orders. 
    \item %\cite{Mazza:2021rgq}
    J.~Mazza, E.~Franzin and S.~Liberati,
    ``A novel family of rotating black hole mimickers,''
    JCAP \textbf{04}, 082 (2021)
    doi:10.1088/1475-7516/2021/04/082
    [arXiv:2102.01105 [gr-qc]].
    135 citations counted in INSPIRE as of 16 Sep 2024\\
    The author employ the Newman-Janis procedure to generalize SV metric, which generates the rotating blackhole mimickers. A method wowards the quantum theory of gravity.
\end{itemize}
\section*{Double copy relation}
\begin{itemize}
    \item %\cite{Johansson:2019dnu}
    H.~Johansson and A.~Ochirov,
    ``Double copy for massive quantum particles with spin,''
    JHEP \textbf{09}, 040 (2019)
    doi:10.1007/JHEP09(2019)040
    [arXiv:1906.12292 [hep-th]].
    161 citations counted in INSPIRE as of 16 Sep 2024\\
    This paper explores the double copy structure of gravitational theories coupled to massive matter with spin, which is relevant for understanding black hole scattering and gravitational waves.
    \item %\cite{Adamo:2022dcm}
    T.~Adamo, J.~J.~M.~Carrasco, M.~Carrillo-Gonz\'alez, M.~Chiodaroli, H.~Elvang, H.~Johansson, D.~O'Connell, R.~Roiban and O.~Schlotterer,
    ``Snowmass White Paper: the Double Copy and its Applications,''
    [arXiv:2204.06547 [hep-th]].
    92 citations counted in INSPIRE as of 16 Sep 2024\\
    This Snowmass white paper provides a brief introduction to the double copy, its applications,
    current research and future challenges.
    \item %\cite{Chacon:2021wbr}
    E.~Chac\'on, S.~Nagy and C.~D.~White,
    ``The Weyl double copy from twistor space,''
    JHEP \textbf{05}, 2239 (2021)
    doi:10.1007/JHEP05(2021)239
    [arXiv:2103.16441 [hep-th]].
    72 citations counted in INSPIRE as of 16 Sep 2024\\
    The author show how
    the current form and scope of the Weyl double copy can be derived from a certain procedure
    in twistor space.
    \item %\cite{Borsten:2021hua}
    L.~Borsten, H.~Kim, B.~Jur\v{c}o, T.~Macrelli, C.~Saemann and M.~Wolf,
    ``Double Copy from Homotopy Algebras,''
    Fortsch. Phys. \textbf{69}, no.8-9, 2100075 (2021)
    doi:10.1002/prop.202100075
    [arXiv:2102.11390 [hep-th]].
    76 citations counted in INSPIRE as of 16 Sep 2024\\
    The author give a detailed exposition of the BRST Lagrangian double copy and show that colour-kinematics duality and the BRST-Lagrangian double copy can be
 elegantly articulated in terms of homotopy algebras.
\end{itemize}
\section*{Gravitational waves}
\begin{itemize}
    \item %\cite{DiVecchia:2023frv}
    P.~Di Vecchia, C.~Heissenberg, R.~Russo and G.~Veneziano,
    ``The gravitational eikonal: From particle, string and brane collisions to black-hole encounters,''
    Phys. Rept. \textbf{1083}, 1-169 (2024)
    doi:10.1016/j.physrep.2024.06.002
    [arXiv:2306.16488 [hep-th]].
    60 citations counted in INSPIRE as of 16 Sep 2024\\
    Review paper. THe author discuss how the gravitational ekinoal approach can be applied invarious different physical
    set ups involving particles,strings and branes and then we mainly concentrate on the most recent
    developments.
\end{itemize}
\section*{Effective theories of BH}
\begin{itemize}
    \item %\cite{Cangemi:2022bew}
    L.~Cangemi, M.~Chiodaroli, H.~Johansson, A.~Ochirov, P.~Pichini and E.~Skvortsov,
    ``Kerr Black Holes From Massive Higher-Spin Gauge Symmetry,''
    Phys. Rev. Lett. \textbf{131}, no.22, 221401 (2023)
    doi:10.1103/PhysRevLett.131.221401
    [arXiv:2212.06120 [hep-th]].
    52 citations counted in INSPIRE as of 16 Sep 2024\\
    The author propose that the dynamics of Kerr black holes is strongly constrained by the principle of
    gauge symmetry, and show that the known three-point Kerr amplitudes are uniquely
    predicted using massive higher-spin gauge symmetry
\end{itemize}
\section*{Classical gravitational scattering versus Quantum part}
\begin{itemize}
    \item %\cite{Aoude:2022thd}
    R.~Aoude, K.~Haddad and A.~Helset,
    ``Classical Gravitational Spinning-Spinless Scattering at O(G2S\ensuremath{\infty}),''
    Phys. Rev. Lett. \textbf{129}, no.14, 141102 (2022)\\
    doi:10.1103/PhysRevLett.129.141102
    [arXiv:2205.02809 [hep-th]].
    73 citations counted in INSPIRE as of 16 Sep 2024\\
    The author calculate
    the classical gravitational scattering amplitude for one spinning and one spinless object at $O(G^2)$
    and all orders in spin, which exhibits the spin structure that has been
    conjectured to describe Kerr black holes
    \item %\cite{Alessio:2022kwv}
    F.~Alessio and P.~Di Vecchia,
    ``Radiation reaction for spinning black-hole scattering,''
    Phys. Lett. B \textbf{832}, 137258 (2022)
    doi:10.1016/j.physletb.2022.137258
    [arXiv:2203.13272 [hep-th]].
    54 citations counted in INSPIRE as of 16 Sep 2024\\
    Starting from the leading soft term of the 5-point amplitude, involving a graviton and two Kerr black holes, the author determine the
    radiative contribution to the real part of the two-loop eikonal.
    \item %\cite{Jakobsen:2022fcj}
    G.~U.~Jakobsen and G.~Mogull,
    ``Conservative and Radiative Dynamics of Spinning Bodies at Third Post-Minkowskian Order Using Worldline Quantum Field Theory,''
    Phys. Rev. Lett. \textbf{128}, no.14, 141102 (2022)
    doi:10.1103/PhysRevLett.128.141102
    [arXiv:2201.07778 [hep-th]].
    96 citations counted in INSPIRE as of 16 Sep 2024\\
    Using the spinning worldline quantum field theory formalism, the author compute the conservative scattering angle and the radiation-reaction effects.
    \item %\cite{Cristofoli:2021jas}
    A.~Cristofoli, R.~Gonzo, N.~Moynihan, D.~O'Connell, A.~Ross, M.~Sergola and C.~D.~White,
    ``The uncertainty principle and classical amplitudes,''
    JHEP \textbf{06}, 181 (2024)
    doi:10.1007/JHEP06(2024)181
    [arXiv:2112.07556 [hep-th]].
    80 citations counted in INSPIRE as of 16 Sep 2024\\
    The author discussed how the classical limit is encoded in the quantum-first definition of field theory through scattering amplitudes
    \item %\cite{Aoude:2021oqj}
    R.~Aoude and A.~Ochirov,
    ``Classical observables from coherent-spin amplitudes,''
    JHEP \textbf{10}, 008 (2021)
    doi:10.1007/JHEP10(2021)008
    [arXiv:2108.01649 [hep-th]].
    75 citations counted in INSPIRE as of 16 Sep 2024\\
    The author promote the KMOC formalism  to describe general classical spinning objects by using coherent spin states. 
\end{itemize}
\end{document}