\documentclass[a4paper,12pt]{article}
\usepackage[margin=1in]{geometry}
\title{}
\author{}
\begin{document}
    \begin{itemize}
        \item %\cite{Arkani-Hamed:2019ymq}
        N.~Arkani-Hamed, Y.~t.~Huang and D.~O'Connell,
        ``Kerr black holes as elementary particles,''
        JHEP \textbf{01}, 046 (2020)
        doi:10.1007/JHEP01(2020)046
        [arXiv:1906.10100 [hep-th]].
        224 citations counted in INSPIRE as of 13 Sep 2024
    \end{itemize}
\section*{Towards Quantum Gravity}
\begin{itemize}
    \item %\cite{Aoude:2020onz}
    R.~Aoude, K.~Haddad and A.~Helset,
    ``On-shell heavy particle effective theories,''
    JHEP \textbf{05}, 051 (2020)
    doi:10.1007/JHEP05(2020)051
    [arXiv:2001.09164 [hep-th]].
    163 citations counted in INSPIRE as of 16 Sep 2024\\
    The author introduce on-shell variables for HPETs with the aim to present such a formalism will allow the extension of
    HBET to higher spins and to facilitate its application to higher loop orders. 
    \item %\cite{Mazza:2021rgq}
    J.~Mazza, E.~Franzin and S.~Liberati,
    ``A novel family of rotating black hole mimickers,''
    JCAP \textbf{04}, 082 (2021)
    doi:10.1088/1475-7516/2021/04/082
    [arXiv:2102.01105 [gr-qc]].
    135 citations counted in INSPIRE as of 16 Sep 2024\\
    The author employ the Newman-Janis procedure to generalize SV metric, which generates the rotating blackhole mimickers. A method towards the quantum theory of gravity.
\end{itemize}
\section*{Double copy relation}
\begin{itemize}
    \item %\cite{Johansson:2019dnu}
    H.~Johansson and A.~Ochirov,
    ``Double copy for massive quantum particles with spin,''
    JHEP \textbf{09}, 040 (2019)
    doi:10.1007/JHEP09(2019)040
    [arXiv:1906.12292 [hep-th]].
    161 citations counted in INSPIRE as of 16 Sep 2024\\
    This paper explores the double copy structure of gravitational theories coupled to massive matter with spin, which is relevant for understanding black hole scattering and gravitational waves.
    \item %\cite{Adamo:2022dcm}
    T.~Adamo, J.~J.~M.~Carrasco, M.~Carrillo-Gonz\'alez, M.~Chiodaroli, H.~Elvang, H.~Johansson, D.~O'Connell, R.~Roiban and O.~Schlotterer,
    ``Snowmass White Paper: the Double Copy and its Applications,''
    [arXiv:2204.06547 [hep-th]].
    92 citations counted in INSPIRE as of 16 Sep 2024\\
    This Snowmass white paper provides a brief introduction to the double copy, its applications,
    current research and future challenges.
    \item %\cite{Chacon:2021wbr}
    E.~Chac\'on, S.~Nagy and C.~D.~White,
    ``The Weyl double copy from twistor space,''
    JHEP \textbf{05}, 2239 (2021)
    doi:10.1007/JHEP05(2021)239
    [arXiv:2103.16441 [hep-th]].
    72 citations counted in INSPIRE as of 16 Sep 2024\\
    The author show how
    the current form and scope of the Weyl double copy can be derived from a certain procedure
    in twistor space.
    \item%\cite{White:2020sfn}
    C.~D.~White,
    ``Twistorial Foundation for the Classical Double Copy,''
    Phys. Rev. Lett. \textbf{126}, no.6, 061602 (2021)
    doi:10.1103/PhysRevLett.126.061602
    [arXiv:2012.02479 [hep-th]].
    90 citations counted in INSPIRE as of 17 Sep 2024\\ 
    The author
    show that a particular incarnation- the Weyl double copy- can be derived using well-established
    ideas from twistor theory.
    \item %\cite{Borsten:2021hua}
    L.~Borsten, H.~Kim, B.~Jur\v{c}o, T.~Macrelli, C.~Saemann and M.~Wolf,
    ``Double Copy from Homotopy Algebras,''
    Fortsch. Phys. \textbf{69}, no.8-9, 2100075 (2021)
    doi:10.1002/prop.202100075
    [arXiv:2102.11390 [hep-th]].
    76 citations counted in INSPIRE as of 16 Sep 2024\\
    The author give a detailed exposition of the BRST Lagrangian double copy and show that colour-kinematics duality and the BRST-Lagrangian double copy can be
 elegantly articulated in terms of homotopy algebras.
    \item %\cite{Emond:2020lwi}
    W.~T.~Emond, Y.~T.~Huang, U.~Kol, N.~Moynihan and D.~O'Connell,
    ``Amplitudes from Coulomb to Kerr-Taub-NUT,''
    JHEP \textbf{05}, 055 (2022)
    doi:10.1007/JHEP05(2022)055
    [arXiv:2010.07861 [hep-th]].
    71 citations counted in INSPIRE as of 17 Sep 2024\\
    The author dentify the amplitudes corresponding to each of these solutions and con rm
    that the amplitudes double-copy when the solutions are related by the classical double copy.
    \item  %\cite{Godazgar:2020zbv}
    H.~Godazgar, M.~Godazgar, R.~Monteiro, D.~Peinador Veiga and C.~N.~Pope,
    ``Weyl Double Copy for Gravitational Waves,''
    Phys. Rev. Lett. \textbf{126}, no.10, 101103 (2021)
    doi:10.1103/PhysRevLett.126.101103
    [arXiv:2010.02925 [hep-th]].
    73 citations counted in INSPIRE as of 17 Sep 2024\\
    The author establish the status of the Weyl double copy relation for radiative solutions of the vacuum
    Einstein equations.
    \item %\cite{Bah:2019sda}
    I.~Bah, R.~Dempsey and P.~Weck,
    ``Kerr-Schild Double Copy and Complex Worldlines,''
    JHEP \textbf{02}, 180 (2020)
    doi:10.1007/JHEP02(2020)180
    [arXiv:1910.04197 [hep-th]].
    62 citations counted in INSPIRE as of 17 Sep 2024\\
    The author use the classical double copy to identify a necessary condition for Maxwell
 theory sources to constitute single copies of Kerr-Schild solutions, and give a parameterization of the corresponding single copies in terms of Li´ enard-Wiechert fields of charges on complex worldlines.
    \item %\cite{Elor:2020nqe}
    G.~Elor, K.~Farnsworth, M.~L.~Graesser and G.~Herczeg,
    ``The Newman-Penrose Map and the Classical Double Copy,''
    JHEP \textbf{12}, 121 (2020)
    doi:10.1007/JHEP12(2020)121
    [arXiv:2006.08630 [hep-th]].
    66 citations counted in INSPIRE as of 17 Sep 2024\\
    The author present a novel map, called Newman-Penrose map, between a certain class
    of real, exact solutions of Einstein s equations and self-dual solutions of the at-space vacuum Maxwell equations.
    \item %\cite{Borsten:2020xbt}
    L.~Borsten and S.~Nagy,
    ``The pure BRST Einstein-Hilbert Lagrangian from the double-copy to cubic order,''
    JHEP \textbf{07}, 093 (2020)
    doi:10.1007/JHEP07(2020)093
    [arXiv:2004.14945 [hep-th]].
    64 citations counted in INSPIRE as of 17 Sep 2024\\
    The author construct the pure gravity BRST Einstein Hilbert Lagrangian, to cubic order, using the BRST convolution product and BCJ double-copy.
    \item  %\cite{Plefka:2019wyg}
    J.~Plefka, C.~Shi and T.~Wang,
    ``Double copy of massive scalar QCD,''
    Phys. Rev. D \textbf{101}, no.6, 066004 (2020)
    doi:10.1103/PhysRevD.101.066004
    [arXiv:1911.06785 [hep-th]].
    56 citations counted in INSPIRE as of 17 Sep 2024\\
    The author construct the gravitational theory emerging from the double-copy of massive scalar quantum
    chromodynamics in general dimensions. 
    \item  %\cite{Easson:2023dbk}
    D.~A.~Easson, G.~Herczeg, T.~Manton and M.~Pezzelle,
    ``Isometries and the double copy,''
    JHEP \textbf{09}, 162 (2023)
    doi:10.1007/JHEP09(2023)162
    [arXiv:2306.13687 [gr-qc]].
    18 citations counted in INSPIRE as of 17 Sep 2024\\
    The authors show that the geodicity of the Kerr-Schild vector and the existence of a timelike Killing vector, are in fact consequences of the vacuum conditions instead of assumptions.
    \item %\cite{Bonezzi:2022bse}
    R.~Bonezzi, C.~Chiaffrino, F.~Diaz-Jaramillo and O.~Hohm,
    ``Gauge invariant double copy of Yang-Mills theory: The quartic theory,''
    Phys. Rev. D \textbf{107}, no.12, 126015 (2023)
    doi:10.1103/PhysRevD.107.126015
    [arXiv:2212.04513 [hep-th]].
    27 citations counted in INSPIRE as of 17 Sep 2024\\
    The author gives an explicit gauge invariant, off-shell and local double copy construction of
    gravity
    \item %\cite{Chawla:2022ogv}
    S.~Chawla and C.~Keeler,
    ``Aligned fields double copy to Kerr-NUT-(A)dS,''
    JHEP \textbf{04}, 005 (2023)
    doi:10.1007/JHEP04(2023)005
    [arXiv:2209.09275 [hep-th]].
    17 citations counted in INSPIRE as of 17 Sep 2024\\
    The author find a double copy relation between Abelian gauge fields and a large class of black
 hole spacetimes with spherical horizon topology known as the Kerr-NUT-(A)dS family.
\end{itemize}
\section*{Gravitational waves}
\begin{itemize}
    \item %\cite{DiVecchia:2023frv}
    P.~Di Vecchia, C.~Heissenberg, R.~Russo and G.~Veneziano,
    ``The gravitational eikonal: From particle, string and brane collisions to black-hole encounters,''
    Phys. Rept. \textbf{1083}, 1-169 (2024)
    doi:10.1016/j.physrep.2024.06.002
    [arXiv:2306.16488 [hep-th]].
    60 citations counted in INSPIRE as of 16 Sep 2024\\
    Review paper. The author discuss how the gravitational ekinoal approach can be applied to invarious different physical
    set ups involving particles,strings and branes and then we mainly concentrate on the most recent
    developments.
    
\end{itemize}
\section*{Effective theories of BH}
\begin{itemize}
    \item %\cite{Cangemi:2022bew}
    L.~Cangemi, M.~Chiodaroli, H.~Johansson, A.~Ochirov, P.~Pichini and E.~Skvortsov,
    ``Kerr Black Holes From Massive Higher-Spin Gauge Symmetry,''
    Phys. Rev. Lett. \textbf{131}, no.22, 221401 (2023)
    doi:10.1103/PhysRevLett.131.221401
    [arXiv:2212.06120 [hep-th]].
    52 citations counted in INSPIRE as of 16 Sep 2024\\
    The author propose that the dynamics of Kerr black holes is strongly constrained by the principle of
    gauge symmetry, and show that the known three-point Kerr amplitudes are uniquely
    predicted using massive higher-spin gauge symmetry.
\end{itemize}
\section*{Classical and Quantum gravitational scattering}
\begin{itemize}
    \item %\cite{Aoude:2022thd}
    R.~Aoude, K.~Haddad and A.~Helset,
    ``Classical Gravitational Spinning-Spinless Scattering at O(G2S\ensuremath{\infty}),''
    Phys. Rev. Lett. \textbf{129}, no.14, 141102 (2022)\\
    doi:10.1103/PhysRevLett.129.141102
    [arXiv:2205.02809 [hep-th]].
    73 citations counted in INSPIRE as of 16 Sep 2024\\
    The author calculate
    the classical gravitational scattering amplitude for one spinning and one spinless object at $O(G^2)$
    and all orders in spin, which exhibits the spin structure that has been
    conjectured to describe Kerr black holes
    \item %\cite{Alessio:2022kwv}
    F.~Alessio and P.~Di Vecchia,
    ``Radiation reaction for spinning black-hole scattering,''
    Phys. Lett. B \textbf{832}, 137258 (2022)
    doi:10.1016/j.physletb.2022.137258
    [arXiv:2203.13272 [hep-th]].
    54 citations counted in INSPIRE as of 16 Sep 2024\\
    Starting from the leading soft term of the 5-point amplitude, involving a graviton and two Kerr black holes, the author determine the
    radiative contribution to the real part of the two-loop eikonal.
    \item %\cite{Jakobsen:2022fcj}
    G.~U.~Jakobsen and G.~Mogull,
    ``Conservative and Radiative Dynamics of Spinning Bodies at Third Post-Minkowskian Order Using Worldline Quantum Field Theory,''
    Phys. Rev. Lett. \textbf{128}, no.14, 141102 (2022)
    doi:10.1103/PhysRevLett.128.141102
    [arXiv:2201.07778 [hep-th]].
    96 citations counted in INSPIRE as of 16 Sep 2024\\
    Using the spinning worldline quantum field theory formalism, the author compute the conservative scattering angle and the radiation-reaction effects.
    \item %\cite{Cristofoli:2021jas}
    A.~Cristofoli, R.~Gonzo, N.~Moynihan, D.~O'Connell, A.~Ross, M.~Sergola and C.~D.~White,
    ``The uncertainty principle and classical amplitudes,''
    JHEP \textbf{06}, 181 (2024)
    doi:10.1007/JHEP06(2024)181
    [arXiv:2112.07556 [hep-th]].
    80 citations counted in INSPIRE as of 16 Sep 2024\\
    The author discussed how the classical limit is encoded in the quantum-first definition of field theory through scattering amplitudes
    \item %\cite{Aoude:2021oqj}
    R.~Aoude and A.~Ochirov,
    ``Classical observables from coherent-spin amplitudes,''
    JHEP \textbf{10}, 008 (2021)
    doi:10.1007/JHEP10(2021)008
    [arXiv:2108.01649 [hep-th]].
    75 citations counted in INSPIRE as of 16 Sep 2024\\
    The author promote the KMOC formalism  to describe general classical spinning objects by using coherent spin states. 
    \item %\cite{Cheung:2020gbf}
    C.~Cheung, N.~Shah and M.~P.~Solon,
    ``Mining the Geodesic Equation for Scattering Data,''
    Phys. Rev. D \textbf{103}, no.2, 024030 (2021)
    doi:10.1103/PhysRevD.103.024030
    [arXiv:2010.08568 [hep-th]].
    65 citations counted in INSPIRE as of 17 Sep 2024\\
    The author explore how the geodesic equation encodes conservative dynamics in the presence of an arbitrary
    perturbative correction away from a non-spinning black hole binary system in general relativity.
    \item %\cite{Luna:2023uwd}
    A.~Luna, N.~Moynihan, D.~O'Connell and A.~Ross,
    ``Observables from the spinning eikonal,''
    JHEP \textbf{08}, 045 (2024)
    doi:10.1007/JHEP08(2024)045
    [arXiv:2312.09960 [hep-th]].
    14 citations counted in INSPIRE as of 17 Sep 2024\\
    The author study the classical dynamics of spinning particles using scattering amplitudes
 and eikonal exponentiation.
    \item %\cite{Kosmopoulos:2023bwc}
    D.~Kosmopoulos and M.~P.~Solon,
    ``Gravitational self force from scattering amplitudes in curved space,''
    JHEP \textbf{03}, 125 (2024)
    doi:10.1007/JHEP03(2024)125
    [arXiv:2308.15304 [hep-th]].
    18 citations counted in INSPIRE as of 17 Sep 2024\\
    The author employ scattering amplitudes in curved space to model the dynamics of a
light probe particle orbiting in the background spacetime.
    \item %\cite{Bern:2023ity}
    Z.~Bern, D.~Kosmopoulos, A.~Luna, R.~Roiban, T.~Scheopner, F.~Teng and J.~Vines,
    ``Quantum field theory, worldline theory, and spin magnitude change in orbital evolution,''
    Phys. Rev. D \textbf{109}, no.4, 045011 (2024)
    doi:10.1103/PhysRevD.109.045011
    [arXiv:2308.14176 [hep-th]].
    21 citations counted in INSPIRE as of 17 Sep 2024\\
    The author clarify the nature of additional Wilson coefficients appear compared to current worldline approaches.
    \item %\cite{Gonzo:2023goe}
    R.~Gonzo and C.~Shi,
    ``Boundary to bound dictionary for generic Kerr orbits,''
    Phys. Rev. D \textbf{108}, no.8, 084065 (2023)
    doi:10.1103/PhysRevD.108.084065
    [arXiv:2304.06066 [hep-th]].
    20 citations counted in INSPIRE as of 17 Sep 2024\\
    The author establish a new relation between classical observables for scattering and bound orbits of a massive probe particle in a Kerr background.
    \item %\cite{Bjerrum-Bohr:2022ows}
    N.~E.~J.~Bjerrum-Bohr, L.~Plant\'e and P.~Vanhove,
    ``Effective Field Theory and Applications: Weak Field Observables from Scattering Amplitudes in Quantum Field Theory,''
    [arXiv:2212.08957 [hep-th]].
    19 citations counted in INSPIRE as of 17 Sep 2024\\ 
    The author discuss classical components
    of perturbative weak-field scattering amplitudes until the fourth post-Minkowskian order based on an effective field theory extension of the Einstein-Hilbert action.

\end{itemize}
\section*{Holography}
\begin{itemize}
    \item %\cite{Pasterski:2020pdk}
    S.~Pasterski and A.~Puhm,
    ``Shifting spin on the celestial sphere,''
    Phys. Rev. D \textbf{104}, no.8, 086020 (2021)
    doi:10.1103/PhysRevD.104.086020
    [arXiv:2012.15694 [hep-th]].
    69 citations counted in INSPIRE as of 17 Sep 2024\\
    The author aims to  expand the existing framework surrounding conformal primary
    states. They explore conformal primary wavefunctions for all half integer spins up to the graviton within the Celestial Holography program.
    \item %\cite{Crawley:2023brz}
    E.~Crawley, A.~Guevara, E.~Himwich and A.~Strominger,
    ``Self-dual black holes in celestial holography,''
    JHEP \textbf{09}, 109 (2023)
    doi:10.1007/JHEP09(2023)109
    [arXiv:2302.06661 [hep-th]].
    19 citations counted in INSPIRE as of 17 Sep 2024\\
    The author use the celestial holography formalism to construct 2D quantum states for 4D linearized rotating self-dual black holes in (2,2) signature Klein space.
\end{itemize}
\section*{Formal Amplitude Method}
\begin{itemize}
    \item  %\cite{Chen:2023dcx}
    A.~S.~K.~Chen, H.~Elvang and A.~Herderschee,
    ``Bootstrapping the String Kawai-Lewellen-Tye Kernel,''
    Phys. Rev. Lett. \textbf{131}, no.3, 031602 (2023)\\
    doi:10.1103/PhysRevLett.131.031602
    [arXiv:2302.04895 [hep-th]].\\
    15 citations counted in INSPIRE as of 17 Sep 2024\\
    This paper extends the minimal rank KLT bootstrap to 6-point, finding novel restrictions on the 4
    and 5-point double-copy maps.
    \item %\cite{Cristofoli:2022phh}
    A.~Cristofoli, A.~Elkhidir, A.~Ilderton and D.~O'Connell,
    ``Large gauge effects and the structure of amplitudes,''
    JHEP \textbf{06}, 204 (2023)
    doi:10.1007/JHEP06(2023)204
    [arXiv:2211.16438 [hep-th]].
    12 citations counted in INSPIRE as of 17 Sep 2024\\
    The author show that large gauge transformations modify the structure of momentum conservation.
\end{itemize}
\end{document}